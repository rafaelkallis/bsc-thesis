\documentclass[xcolor={dvipsnames}]{beamer}
%\documentclass[handout]{beamer} %to produce the handout

\usepackage{beamerthemeshadow}
\usepackage{graphicx}
\usepackage{textcomp}
\usepackage{xyling}
\usepackage{stmaryrd}
\usepackage{multirow}
\usepackage{pdfpages}
\usepackage[center]{caption}
\usepackage{forest}
\usepackage{subcaption}
\usepackage{framed}

%\usetikzlibrary{decorations.pathreplacing}

\title{An Adaptive Index for Hierarchical Database Systems}
\titlegraphic{\includegraphics[height=.8cm]{IFIlogo.eps}}
\author{Rafael Kallis}
\institute{BSc Thesis}

\begin{document}
    \bibliographystyle{acm}
\frame{\titlepage}

%%%%%%%%%%%%%%%%%%%%%%%%%%%%%%%%%%%%%%%%%%%%%%%%%%%%%%%%%%%%%%%%%%%%%%%%%%%%%%%%%%%%%%%%%%%%%%%%%%%%%%%
%%%%%%%%%%%%%%%%%%%%%%%%%%%%%%%%%%%%%%%%%%%%%%%%%%%%%%%%%%%%%%%%%%%%%%%%%%%%%%%%%%%%%%%%%%%%%%%%%%%%%%%
%%%%%%%%%%%%%%%%%%%%%%%%%%%%%%%%%%%%%%%%%%%%%%%%%%%%%%%%%%%%%%%%%%%%%%%%%%%%%%%%%%%%%%%%%%%%%%%%%%%%%%%

\section{Introduction}
\subsection{Abstract \& Outline}
\frame{
    \begin{Large}
        The Workload-Aware Property Index (WAPI):
    \end{Large}
    
    \begin{itemize}
        \item Detects frequently updated nodes
        \item Stops pruning such volatile nodes
        \item Significantly improves update throughput
    \end{itemize}
}
\frame{
    \begin{Large}
        Unproductive Nodes are an unwanted byproduct:
    \end{Large}

    \begin{itemize}
        \item When the workload changes, volatile nodes cease to be volatile
        \item Then they waste space and slow down queries
        \item They do not contribute to a query match and contain no data
    \end{itemize}
}
\frame{
    \begin{Large}
        In this thesis we:
    \end{Large}

    \begin{itemize}
        \item Design and implement two solutions in order to mitigate unproductive nodes
        \item Analyze the factors impacting the production of unproductive nodes
        \item Empirically evaluate and compare our two solutions
    \end{itemize}
}
\subsection{CMS Workload}
\frame{
    \begin{large}
        A Content Management System's (CMS) workload is:
    \end{large}
    
    \begin{itemize}
        \item skewed
        \item update-heavy
        \item changing
    \end{itemize}

    %CMSs frequently use a job-queuing system that has the noted characteristics.
}
\frame{
    Skewed Workload: small subset of nodes gets frequently updated 

    \vspace{3cm}
    \begin{figure}[H]
        \centering
        \begin{subfigure}{0.30\textwidth}
            \centering
            \scriptsize
            \begin{forest}
                [,circle,draw,fill=YellowOrange
                [,circle,draw,fill=YellowOrange
                [,circle,draw,fill=YellowOrange
                ]
                [,circle,draw,fill=YellowOrange
                [,circle,draw,fill=YellowOrange]
                [,phantom]
                ]
                ]
                [,circle,draw,fill=YellowOrange
                [,phantom]
                [,circle,draw,fill=YellowOrange
                [,circle,draw,fill=YellowOrange]
                [,circle,draw,fill=YellowOrange]
                ]
                ]
                ]
            \end{forest}

            No skew
        \end{subfigure}
        \begin{subfigure}{0.30\textwidth}
            \centering
            \scriptsize
            \begin{forest}
                [,circle,draw,fill=Yellow
                [,circle,draw,fill=Yellow
                [,circle,draw,fill=Yellow
                ]
                [,circle,draw,fill=Yellow
                [,circle,draw,fill=Orange]
                [,phantom]
                ]
                ]
                [,circle,draw,fill=Orange
                [,phantom]
                [,circle,draw,fill=Yellow
                [,circle,draw,fill=Red]
                [,circle,draw,fill=Orange]
                ]
                ]
                ]
            \end{forest}

            Normal skew
        \end{subfigure}
        \begin{subfigure}{0.30\textwidth}
            \centering
            \scriptsize
            \begin{forest}
                [,circle,draw,fill=Yellow
                [,circle,draw,fill=Yellow
                [,circle,draw,fill=Yellow
                ]
                [,circle,draw,fill=Yellow
                [,circle,draw,fill=Yellow]
                [,phantom]
                ]
                ]
                [,circle,draw,fill=Yellow
                [,phantom]
                [,circle,draw,fill=Yellow
                [,circle,draw,fill=Red]
                [,circle,draw,fill=Red]
                ]
                ]
                ]
            \end{forest}

            High skew
        \end{subfigure}
    \end{figure}
}
\frame{
    Changing workload: as time passes, hotspots change

    \vspace{3cm}
    \begin{figure}[H]
        \centering
        \begin{subfigure}{0.20\textwidth}
            \centering
            \scriptsize
            \begin{forest}
                [,circle,draw,fill=Yellow
                [,circle,draw,fill=Yellow
                [,circle,draw,fill=Orange
                ]
                [,circle,draw,fill=Yellow
                [,circle,draw,fill=Red]
                [,phantom]
                ]
                ]
                [,circle,draw,fill=Yellow
                [,phantom]
                [,circle,draw,fill=Yellow
                [,circle,draw,fill=Yellow
                [,circle,draw,fill=Yellow]
                [,circle,draw,fill=Yellow]
                ]
                [,circle,draw,fill=Orange]
                ]
                ]
                ]
            \end{forest}
        \end{subfigure}
        \begin{subfigure}{0.10\textwidth}
            \centering
            $\longrightarrow$
        \end{subfigure}
        \begin{subfigure}{0.20\textwidth}
            \centering
            \scriptsize
            \begin{forest}
                [,circle,draw,fill=Yellow
                [,circle,draw,fill=Yellow
                [,circle,draw,fill=Yellow
                ]
                [,circle,draw,fill=Yellow
                [,circle,draw,fill=Yellow]
                [,phantom]
                ]
                ]
                [,circle,draw,fill=Orange
                [,phantom]
                [,circle,draw,fill=Yellow
                [,circle,draw,fill=Yellow
                [,circle,draw,fill=Orange]
                [,circle,draw,fill=Red]
                ]
                [,circle,draw,fill=Orange]
                ]
                ]
                ]
            \end{forest}
        \end{subfigure}
        \begin{subfigure}{0.10\textwidth}
            \centering
            $\longrightarrow$
        \end{subfigure}
        \begin{subfigure}{0.20\textwidth}
            \centering
            \scriptsize
            \begin{forest}
                [,circle,draw,fill=Yellow
                [,circle,draw,fill=Yellow
                [,circle,draw,fill=Orange
                ]
                [,circle,draw,fill=Yellow
                [,circle,draw,fill=Yellow]
                [,phantom]
                ]
                ]
                [,circle,draw,fill=Yellow
                [,phantom]
                [,circle,draw,fill=Yellow
                [,circle,draw,fill=Red
                [,circle,draw,fill=Orange]
                [,circle,draw,fill=Yellow]
                ]
                [,circle,draw,fill=Yellow]
                ]
                ]
                ]
            \end{forest}
        \end{subfigure}
    \end{figure}
}
\frame{
    CMSs usually use a job-queuing system that has the notes characteristics
}
\subsection{Workload-Aware Property Index}
\frame{
    Hierarchical Database with WAPI

    \begin{figure}[t]
        \centering
        \scriptsize
        \begin{forest}
            [
            [$\lambda:i$
            [$\lambda:\text{pub}$
            [$\lambda:\text{now}$
            [$\lambda:a$
            [$\lambda:b$
            [$\lambda:d$ \\ $\text{pub}:\text{now}$, align=center, base=bottom, name=i_node] {
                \draw[<-,gray] (.north west)--++(-5em,1em)
                node[anchor=east]{\textit{Matching Node}};
            }
            ]
            [,phantom]
            ]
            ]
            ]
            ] {
                \draw[<-,gray] (.south west)--++(-5em, -1em)
                node[anchor=east]{\textit{Index Subtree Root}};
            }
            [$\lambda:a$
            [$\lambda:b$
            [$\lambda:d$ \\ $\text{pub}:\text{now}$, align=center, base=bottom, name=c_node]
            ]
            [$\lambda:c$
            [$\lambda:e$]
            ]
            ] {
                \draw[<-,gray] (.south east)--++(5em, -1em)
                node[anchor=west]{\textit{Content Subtree Root}};
            }
            ]
            \draw[->,dotted] (i_node) to[out=east, in=south] node[gray,midway,right]{\textit{corresponding content node}} (c_node);
        \end{forest}
    \end{figure}
}
\frame{
    Oak mostly executes content-and-structure (CAS) queries~\cite{CM15}.
    We denote node $n$'s property $k$ as $n[k]$ and node $n$'s descendants as
    $desc(n)$.

    \begin{definition}[CAS Query]
        Given node $m$, property $k$ and value $v$, a CAS query
        $Q(k,v,m)$ returns all descendants of $m$ which have $k$ set to $v$, i.e.,
        $$ Q(k,v,m) = \{ n | n \in desc(m) \land n[k] = v\} $$
        \label{def:cas-query}
    \end{definition}
}
\frame{
    Volatility is the measure which is used by the WAPI in order to distinguish
    whether to remove a node or not from the index.
    Wellenzohn et al.~\cite{KW17} propose to look at the recent transactional
    workload to check whether a node $n$ is volatile.
}
\frame{
    \begin{definition}[Volatility Count]
        The volatility count $vol(n)$ of index node $n$ on database instance $O_i$, is the number of
        times node $n$ was added or removed from snapshots contained in a Sliding
        Window of Length $L$ over history $H_i$, i.e.,
        \vspace{3mm}
        \begin{align*}
            vol(n) = | \{ G^b | G^b \in H_i \land t(G^b) \in [t_n-L+1, t_n] \land \exists G^a[ \\
                \qquad G^a = pre(G^b) \land ([n^a \notin N(G^a) \land n^b \in N(G^b)]\lor \\
            \qquad [n^a \in N(G^a) \land n^b \notin N(G^b)] )]\} |
        \end{align*}
    \end{definition}
}
\frame{
    \begin{definition}[Volatile Node]
        Index node $n$ is volatile iff $n$'s volatility count is greater or equal than 
        the volatility threshold $\tau$, i.e.,
        \vspace{3mm}
        $$ volatile(n) \iff vol(n) \geq \tau $$
    \end{definition}
}
\frame{
    \begin{figure}[H]
        \centering
        \begin{large}
            $$ G^0 \xrightarrow{\quad T_1 \quad} G^1 \xrightarrow{\quad T_2 \quad} G^2$$
        \end{large}
        \begin{subfigure}{0.3\textwidth}
            \centering \tiny{
                \begin{framed}
                    \begin{forest}
                        [
                        [$\lambda$:$i$
                        [,phantom]
                        ]
                        [,phantom]
                        [,phantom]
                        [$\lambda$:$a$
                        [$\lambda$:$b$
                        [$\lambda$:$d$]
                        ] 
                        [,phantom]
                        [$\lambda$:$c$
                        [$\lambda$:$e$]
                        ]
                        ]
                        ]
                    \end{forest}

                    \vspace{22mm}
                \end{framed}
                } \footnotesize{ Snapshot $G^0$

            $t(G^0) = t$ }
        \end{subfigure}
        \begin{subfigure}{0.3\textwidth}
            \centering \tiny{
                \begin{framed}
                    \begin{forest}
                        [
                        [$\lambda$:$i$
                        [$\lambda$:pub, color=RoyalBlue
                        [$\lambda$:now, color=RoyalBlue
                        [$\lambda$:$a$, color=RoyalBlue
                        [$\lambda$:$b$, color=RoyalBlue
                        [$\lambda$:$d$ \\ pub:now, color=RoyalBlue, align=center, base=bottom]
                        ]
                        [,phantom]
                        ]
                        ]
                        ]
                        ]
                        [$\lambda$:$a$
                        [$\lambda$:$b$
                        [$\lambda$:$d$ \\ pub:now, align=center, base=bottom]
                        ]
                        [$\lambda$:$c$
                        [$\lambda$:$e$]
                        ]
                        ]
                        ]
                    \end{forest}
                \end{framed}
                } \footnotesize{ Snapshot $G^1$

            $t(G^1) = t+1$ }
        \end{subfigure}
        \begin{subfigure}{0.3\textwidth}
            \centering \tiny{
                \begin{framed}
                    \begin{forest}
                        [
                        [$\lambda$:$i$
                        [$\lambda$:pub, color=RoyalBlue
                        [$\lambda$:now, color=RoyalBlue
                        [$\lambda$:$a$, color=RoyalBlue
                        [$\lambda$:$b$, color=RoyalBlue
                        [$\lambda$:$d$, color=RoyalBlue]
                        ]
                        [,phantom]
                        [,phantom]
                        ]
                        ]
                        ]
                        ]
                        [$\lambda$:$a$
                        [$\lambda$:$b$
                        [$\lambda$:$d$]
                        ]
                        [,phantom]
                        [$\lambda$:$c$
                        [$\lambda$:$e$]
                        ]
                        ]
                        ]
                    \end{forest}

                    \vspace{3mm}
                \end{framed}
                } \footnotesize{ Snapshot $G^2$

            $t(G^2) = t+2$ }
        \end{subfigure}
    \end{figure}
}
\section{Unproductive Nodes}
\subsection{Subsection 1}
\frame{
    \frametitle{frame title}
    \begin{block}{Block title}
        Block contents
    \end{block}
    \begin{itemize}
        \item item1
        \item item2
    \end{itemize}
}
\section{GC \& QTP}
\section{Conclusion}
\subsection{References}
\frame{
    \frametitle{References}
    \scriptsize
    \bibliography{bib}
}

\end{document}
